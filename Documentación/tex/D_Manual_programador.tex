\apendice{Documentación técnica de programación}

\section{Introducción}
Esta sección es para otros desarroladores, para que en un futuro puedan continuar nuestro proyecto y entenderlo. Se describen el funcionamiento del proyecto, y que aspectos se podrían mejorar o modificar.

\section{Estructura de directorios}
Nuestro proyecto se divide en dos partes, una donde tendremos nuestro código fuente, y otra con la documentación.

\subsection{Documentación}
En esta carpeta es donde podemos encontrar la documentación de la memoria y anexos.
\begin{itemize}
	\item img: Esta carpeta contiene las distintas imágenes utilizadas en la memoria y anexos.
	\item tex: Están las distintas partes en las que se dividen la memoria y los anexos.
	\item Los pdfs de la memoria, anexos y la bibliografía.
\end{itemize}

\subsection{Src}
Esta carpeta contiene los ficheros código fuente del proyecto.
\begin{itemize}
	\item DecisionTreeClassifier vs Ensembles: Notebook que muestra la comparación de los 3 algoritmos realizados con el DecisionTreeClassifier.
	\item Example of base classifiers: Notebook en el que podemos ejecutar los clasificadores base.
	\item Example of ensembles classifiers: Notebook donde ejecutamos los ensembles de cada uno de los clasficadores base.
	\item Example with real ML data set: Notebook donde probamos la ejecución de nuestros clasificadores en un conjunto de datos reales.
	\item Graphics: Notebook en el que podemos ver las gráficas de la comparación de los 3 clasificadores base realizados, en los que se puede ver como se dividen los datos.
	\item flags: Fichero con los datos reales.
	\item sklearn\_ubu: Esta carpeta contiene los ficheros  con los códigos fuente de los algoritmos, sus clasificadores base y sus ensembles:
	\begin{itemize}
		\item base\_disturbing\_neighbors
		\item base\_random\_oracles
		\item base\_rotation\_forest
		\item disturbing\_neighbors
		\item random\_oracles
		\item rotation\_forest
		\item homogeneous\_ensemble
	\end{itemize}
\end{itemize}

\section{Manual del programador}
En esta sección vamos a describir como instalar las diferentes herramientas necesarias para realizar el proyecto.

\subsection{SonarQube}
Para analizar la calidad del código se ha analizado mediante la herramienta web de SonarQube.
Si queremos comprobar nuestro código con esta herramienta hay que seguir una serie de pasos:
\begin{itemize}
	\item Entrar en https://www.sonarqube.org/, podemos elegir entre descargarla o usar online, nosotros elegiremos esta segunda, que nos redirigirá a la página de https://about.sonarcloud.io/. Para poder utilizarla necesitamos loguearnos, para ello podemos hacerlo con nuestra cuenta de GitHub.
	\item Una vez estemos dentro, en la cabecera clickamos en el icono de la $?$, que nos abrirá una ventana emergente. Y en el menú de la izquierda, pinchamos en tutorials, y dentro en el link de analizar un nuevo proyecto.
	\item En la primera opción elegiremos la opción por defecto de organización personal, porque hay está el proyecto que queremos analizar.
	\item En la segunda opción para generar el token, ponemos un nombre cualquiera.
	\item En la siguiente opción elegiremos el lenguaje y sistema operativo utilizados, en nuestro caso Python y Windows, y ponemos una clave única.
	\item Por último necesitamos descargar un pequeño archivo, lo añadiremos el bin al $PATH$.
	\item Para acabar deberemos entrar a la consola a la ubicación donde se encuentre nuestro proyecto, copiaremos el comando que nos ha creado en la página y lo pegamos en la consola. Esto analizará nuestro código y ya sabremos si tenemos un buen código o necesitamos modificarlo.
\end{itemize}

Nuestro proyecto tiene una  calidad de A, ya que no tiene errores o código duplicado, solo tenemos unas advertencias en que algunos nombres de las variables, no son los adecuados. Aunque la guía de estilos de Python y Pep que hemos seguido si que considera válidos esos nombres, por eso no los cambiaremos.

\section{Compilación, instalación y ejecución del proyecto}

\section{Pruebas del sistema}
