\capitulo{2}{Objetivos del proyecto}

En este apartado se explica cuales son los objetivos que se persiguen con la realización del proyecto, y que han motivado a su realización.

\section{Objetivos}
A continuación se muestra el esquema con todos los puntos a tratar en este proyecto.
\begin{itemize}
\item Implementar el algoritmo Disturbing Neighbors en Scikit-learn:
	\begin{itemize}
		\item Que sirva para datos Single-Label o Multi-Label
		\item Crear el método \texttt{fit} para entrenar un conjunto de datos.
		\item Crear el método \texttt{predict} para predecir según el entrenamiento de unos datos.
		\item Crear el método \texttt{predict\_proba} para predecir probabilidades según el entrenamiento de unos datos.
		\item Probar que la clase funciona correctamente.
	\end{itemize}
	
\item Implementar el algoritmo Random Oracles en Scikit-learn:
	\begin{itemize}
		\item Que sirva para datos Single-Label o Multi-Label
		\item Crear el método \texttt{fit} para entrenar un conjunto de datos.
		\item Crear el método \texttt{predict} para predecir según el entrenamiento de unos datos.
		\item Crear el método \texttt{predict\_proba} para predecir probabilidades según el entrenamiento de unos datos.
		\item Probar que la clase funciona correctamente.
	\end{itemize}
	
\item Implementar el algoritmo Rotation Forest en Scikit-learn:
	\begin{itemize}
		\item Que sirva para datos Single-Label o Multi-Label
		\item Crear el método \texttt{fit} para entrenar un conjunto de datos.
		\item Crear el método \texttt{predict} para predecir según el entrenamiento de unos datos.
		\item Crear el método \texttt{predict\_proba} para predecir probabilidades según el entrenamiento de unos datos.
		\item Probar que la clase funciona correctamente.
	\end{itemize}
\item Crear un notebook para mostrar:
	\begin{itemize}
		\item Los resultados en notebooks de jupyter de los distintos algoritmos.
		\item Usar una semilla para que los datos no cambien y poder compararlos.
		\item Mostrar en un árbol la clasificación del conjunto.
		\item Mostrar gráficas para comparar como se dividen los datos y ver cual de los algoritmos es mejor en cada caso.
		\item Mostrar resultados al usar validación cruzada.
	\end{itemize}
\end{itemize}
