\capitulo{2}{Objetivos del proyecto}

En este apartado,  se explica los objetivos que se quieren conseguir al final, la meta que pretendemos lograr y los motivos que me han llevado a realizar este proyecto.

\section{Objetivos}
A continuación se muestra el esquema con todos los puntos a tratar en este proyecto.
\begin{itemize}
\item Implementar los algoritmos Disturbing Neighbors~\cite{disturbingneighbors}, Random Oracles~\cite{randomoracles} y Rotation Forest~\cite{rotationforest} en Scikit-learn:
	\begin{itemize}
		\item Que sirva para datos Single-Label o Multi-Label.
		\item Crear el método \texttt{fit} para entrenar un conjunto de datos.
		\item Crear el método \texttt{predict} para predecir según el entrenamiento de unos datos.
		\item Crear el método \texttt{predict\_proba} para predecir probabilidades según el entrenamiento de unos datos.
		\item o	Evaluar el correcto funcionamiento de la clases.
	\end{itemize}
	
\item Crear diversos notebooks para mostrar el funcionamiento de los algoritmos:
	\begin{itemize}
		\item Posibilidad de seleccionar el algoritmo.
		\item Permitir utilizar conjuntos reales.
		\item Mostrar su funcionamiento mediante el dibujado del árbol de decisión.
		\item Mostrar el funcionamiento mediante gráficas en dos dimensiones con conjuntos de datos de <<juguete>>.
		\item Mostrar resultados al usar validación cruzada.
	\end{itemize}
\end{itemize}
