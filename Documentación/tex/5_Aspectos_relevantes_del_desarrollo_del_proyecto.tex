\capitulo{5}{Aspectos relevantes del desarrollo del proyecto}

\section{Formación}
El proyecto requería unos conocimientos técnicos de los que desconocía en un principio. Entre ellos, conocimientos sobre Minería de datos, Scikit-Learn y documentar en \LaTeX. Para aprender estos conocimientos se ha necesitado leer documentos científicos, descubriendo la utilidad de ellos, ya que nunca había hecho uso de ellos.

Fue necesario instruirse en cómo implementar algoritmos, para ello tomamos de ejemplo, los algoritmos ya implementados en Scikit-Learn.

Otro de los conocimientos adquiridos en la realización de ese proyecto ha sido la librería Scikit-Leran, porque nosotros queremos implementar algoritmos en dicha librería, porque vamos a tratar la minería de datos, y como queremos clasificar un conjunto de datos para después de entrar poder predecir unos resultados con la mayor precisión posible, Sickit-Learn es una librería perfecta para esto. 

\section{Calidad del Software}
Cualquier persona con pocos conocimientos en minería de datos o programación en Python, puede probarla ya que se han generado unos notebooks para ello, por lo que podremos ejecutarlos teniendo mínimos conocimientos en el tema.

Se ha conseguido una buena calidad de código, ya que se ha comprobado en SonarQube, y tenemos la mejor nota posible. Para tener código claro y fácil de entender se ha ido comentado los métodos/funciones para ver lo que hacer cada parte del código. Se ha seguido la guía de estilo de Python y Pep.