\capitulo{1}{Introducción}

La minería de datos es un campo de 	las ciencias de la computación, que consisten en el análisis de grandes cantidades de datos para descubrir patrones.
Para ello se utiliza el aprendizaje automático, éste pertenece a un subcampo de las ciencias de computación y de la rama de inteligencia artificial, el objetivo del aprendizaje automático es desarrollar unas técnicas que permitan que las máquinas aprendan~\cite{wiki:datamining}.
Dentro de este aprendizaje se encuentra el aprendizaje supervisado, en él normalmente los conjuntos de datos suelen tener solo una variable a predecir, conocido como single-label, pero apareció el Multi-Label, este hace referencia a los conjuntos de datos en lo que cada elemento de la base de datos puede pertenecer a más de una clase, como por ejemplo en el etiquetado de imágenes: en el que una imagen puede tener a la vez las etiquetas <<árbol>>, <<montaña>> y <<mar>>.

En este proyecto vamos a tratar de implementar diversos algoritmos de multi-clasificadores (ensembles), para Multi-Label sobre la librería Scikit-Learn~\cite{scikitlearn} de Python~\cite{python}. Se ha seguido la guía de estilo de Python (PeP~\cite{pep}) y Sklearn (Scikit-Learn). Para que se entienda mejor y sea más gráfico, se han dibujado árboles y gráficas, mostrando los resultados al ejecutar dichos algoritmos sobre un conjunto de datos. Los algoritmos en los que nos vamos a centrar son Disturbing Neighbors~\cite{disturbingneighbors}, Random Oracles~\cite{randomoracles} y Rotation Forest~\cite{rotationforest}.

Vamos a tratar a lo largo del documento de los ensembles, para que estos sean precisos necesitan que los clasificadores base predigan correctamente la clase de las mismas instancias.
Tienen que ser diferentes para complementarse entre ellos, por ello la diversidad es importante. 
¿Cómo puede un ensemble de clasificadores base que han sido generados por el mismo algoritmo tener distintas salidas? Una de las estrategias que podemos usar para ello son los ensembles homogéneos, es decir, mismo algoritmo entrenador con distinto conjunto de datos.
