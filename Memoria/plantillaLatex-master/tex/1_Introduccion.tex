\capitulo{1}{Introducción}

Scikit-learn es una librería de aprendizaje de software libre en el lenguaje de programación de Python. Es una herramienta de las más utilizadas para la minería de datos y el análisis de datos. 

Esta basada en el aprendizaje automático, para ello se consideran un conjunto de n muestras y se intenta predecir las propiedades de los datos desconocidos. Podemos separar los problemas de aprendizaje en dos:
-Aprendizaje supervisado: En el que los datos vienen con atributos adicionales que queremos predecir. Dos de las tareas mas comunes son la clasificación y la regresión. En las de clasificación el programa debe aprender a predecir en que categoría o clase irán los nuevos datos según las nuevas observaciones, como sería predecir si el precio de una acción bajara o subirá. En lo de regresión el programa debe predecir el valor de una variable de respuesta continua, como sería predecir las ventas de un nuevo producto. 

-Aprendizaje no supervisado: Consiste en agrupar observaciones relaciones,dentro de los datos del entrenamiento. Clustering es la que más se utiliza para explorar un conjunto de datos.

